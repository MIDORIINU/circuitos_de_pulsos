En esta sección se explican conceptualmente las decisiones de diseño de nuestro amplificador, se citan antecedentes investigados y se justifican cualitativamente algunas de las elecciones circuitales que se hicieron.
El diseño de un amplificador de tensión como un solo bloque que cumpla con las especificaciones, es una tarea de muy alta complejidad, pero se simplifica enormemente con el uso de técnicas de realimentación, comunes en la teoría de control, que se implementaron en este amplificador. 

%\begin{figure}[H]
%	\centering
%	\includegraphics[width=0.5\textwidth]{img/realimentacion-negativa-bloque}
%	\caption{Modelo general de realimentación negativa.}
%	\label{fig:ampli_feedback}
%\end{figure}


\subsection{Realimentación global}
